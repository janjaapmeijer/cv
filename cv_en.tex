%----------------------------------------------------------------------------------------
%	PACKAGES AND OTHER DOCUMENT CONFIGURATIONS
%----------------------------------------------------------------------------------------

\documentclass[11pt,a4paper,sans]{moderncv}
\moderncvstyle{classic}
\moderncvicons{awesome}
\moderncvcolor{blue}
\usepackage[scale=0.8]{geometry}
\setlength{\hintscolumnwidth}{4cm}

%\usepackage{filecontents}
%\begin{filecontents}{\jobname.bib}
%	@article{A01,
%		author = {Author, A.},
%		journal = {Good Journal},
%		year = {2001},
%		title = {Alpha},
%	}
%\end{filecontents}

% Widen the quote section
\let\originalrecomputecvlengths\recomputecvlengths
\renewcommand*{\recomputecvlengths}{%
	\originalrecomputecvlengths%
	\setlength{\quotewidth}{0.85\textwidth}}

% Reformat cvcomputer entry
%\renewcommand{\cvcomputer}[2]{\cvline{#1}{\small#2}}

% Format references with icons
\newcommand{\cvreference}[9]{%
	\textbf{#1}\newline% Name
	\ifthenelse{\equal{#2}{}}{}{#2\newline}% Function
	\ifthenelse{\equal{#3}{}}{}{#3\newline}% Institute
	\ifthenelse{\equal{#4}{}}{}{\addresssymbol~#4\newline}%
	\ifthenelse{\equal{#5}{}}{}{#5\newline}%
	\ifthenelse{\equal{#6}{}}{}{#6\newline}%
	\ifthenelse{\equal{#7}{}}{}{\emailsymbol~\texttt{\href{mailto:#7}{\nolinkurl{#7}}}\newline}%
	\ifthenelse{\equal{#8}{}}{}{\phonesymbol~#8\newline}
	\ifthenelse{\equal{#9}{}}{}{\mobilephonesymbol~#9}}

% single colum cvcomputer
\renewcommand{\cvcomputer}[2]{\cvline{#1}{\small#2}}


%----------------------------------------------------------------------------------------
%	NAME AND CONTACT INFORMATION SECTION
%----------------------------------------------------------------------------------------

% personal data
\name{Jan Jaap}{Meijer}
\title{Curriculum Vitae}
% \address{135 Waterworks Rd}{Dynnyrne TAS 7005}{Australia}
%\address{Prinses Beatrixstraat 19}{7271 GG Borculo}{the Netherlands}
\phone[mobile]{\href{tel:0061435503003}{+61 435 503 003}}
%\phone[mobile]{\href{tel:0031641709565}{+31 6 417 09 565}}
\email{janjaapmeijer@gmail.com}
\homepage{atsea.science}
\social[linkedin]{janjaapmeijer}
\social[github]{janjaapmeijer}
\social[twitter]{janjaapatsea}
% \extrainfo{\skypesocialsymbol \href{skype:meijerjanjaap?call}{meijerjanjaap}} %
\photo[64pt][0.4pt]{pictures/photo}
\quote{\raggedright{I am an enthusiastic and motivated observational oceanographer who is deeply connected to the ocean not only via my degrees in Atmospheric and Oceanic sciences, but also via my personal interests in surfing and sailing. During my degree I successfully combined my studies with internships and part time jobs, journeys and other commitments showing myself to be self-motivated, organised and able to work productively with others. I have a clear, open mind and use my communication and numeracy skills to solve societal and environmental problems for a sustainable future.}}

% bibliography adjustements (only useful if you make citations in your resume, or print a list of publications using BibTeX)
%   to show numerical labels in the bibliography (default is to show no labels)
%\makeatletter\renewcommand*{\bibliographyitemlabel}{\@biblabel{\arabic{enumiv}}}\makeatother
%\renewcommand*{\bibliographyitemlabel}{[\arabic{enumiv}]}
%   to redefine the bibliography heading string ("Publications")
%\renewcommand{\refname}{Articles}

% bibliography with mutiple entries
%\usepackage{multibib}
%\newcites{articles,book,misc}{{Books},{Others}}

%----------------------------------------------------------------------------------------

\begin{document}
\makecvtitle

%----------------------------------------------------------------------------------------
%	PERSONAL DATA
%----------------------------------------------------------------------------------------

% \section{Personal Data}
% \cvitem{Date of birth}{3 March 1986}
% \cvitem{Place of birth}{Winterswijk, the Netherlands}
% \cvitem{Nationality}{Dutch}
%\cvitem{Marital status}{Single}
%\cvitem{Sex}{Male}

%----------------------------------------------------------------------------------------
%	EDUCATION SECTION
%----------------------------------------------------------------------------------------

\section{Education}
\cventry{2017 -- 2022}{PhD in Quantitative Marine Science}{\textsc{University of Tasmania}}{Hobart, Australia}{}{{Thesis: Meander dynamics in the Antarctic Circumpolar Current.}\\{Supervisors: A/Prof. Dr. H.E. Phillips, Prof. Dr. N.L. Bindoff, Dr. S.R. Rintoul and Dr. A. Foppert}}

\cventry{2014}{Master of Science in Meteorology, Physical Oceanography and Climate}{\textsc{Utrecht University}}{the Netherlands}{}{{\href{https://dspace.library.uu.nl/handle/1874/294251}{Thesis: Modelling the formation of coral cays on platform reefs}}\\{Supervisors: Prof. dr. H.E. de Swart, dr. M. van der Vegt and dr. ir. A. van Dongeren}}

\cventry{2012}{Visiting student at the department of Physical Oceanography}{\textsc{Royal Netherlands Institute for Sea Research (NIOZ)}}{Texel, the Netherlands}{}{}

\cventry{2012}{Visiting student at the department of Coastal Engineering}{\textsc{Delft University of Technology}}{the Netherlands}{}{}

\cventry{2011}{Bachelor of Engineering in Aeronautical Engineering}{\textsc{Inholland University}}{Delft, the Netherlands}{}{}

% \section{PhD Topic}
% %\cvitem{Title}{\emph{}}
% \cvitem{Supervisors}{A/Prof. Dr. H.E. Phillips, Prof. Dr. N.L. Bindoff, Dr. S.R. Rintoul and Dr. A. Foppert}
% \cvitem{Description}{An observational study of the role of standing meanders in slowing the Antarctic Circumpolar Current and transporting heat to Antarctica.}

% \section{Masters Thesis}
% \cvitem{Title}{\emph{Modelling the formation of coral cays on platform reefs.}}
% \cvitem{Supervisors}{Prof. dr. H.E. de Swart, dr. M. van der Vegt and dr. ir. A. van Dongeren}
% \cvitem{Description}{A modelling study to investigate the effects of wave dissipation, wave-generated currents and sediment transport on the formation and positioning of coral cays on platform reefs.}

%----------------------------------------------------------------------------------------
%	WORK EXPERIENCE SECTION
%----------------------------------------------------------------------------------------

\section{Experience}
\subsection{Research experience}

\cventry{2021 -- Present}{Research Assistant}{\textsc{Institute for Marine and Antarctic Studies (IMAS)}}{Hobart, Australia}{\footnotesize{\url{https://www.imas.utas.edu.au}}}{Developing a toolbox for the quality control and data analysis of EM-APEX floats (autonomous oceanographic instrument for measurements of temperature, salinity and velocity in the water column). This toolbox is made available on GitHub, \footnotesize{\url{https://github.com/janjaapmeijer/emapex}}}

\cventry{2019}{Guest investigator}{\textsc{Woods Hole Oceanographic Institute (WHOI)}}{Woods Hole, MA, USA}{\footnotesize{\url{https://www.whoi.edu}}}{Research collaboration with Kurt Polzin (WHOI), Kathleen Donohue, Karen Tracey, Randolph Watts (University of Rhode Island) and supervisory team on “How does topography brake the Antarctic Circumpolar Current?”}

\cventry{2015 -- 2016}{Research Associate}{\textsc{Stockholm Environment Institute (SEI)}}{Stockholm, Sweden}{\footnotesize{\url{http://www.sei-international.org}}}{A case study in which a set of coastal risk tools developed in the \href{http://www.risckit.eu}{RISC-KIT} project are applied, to assess the risk to the Kristianstad Municipality coastline in Sweden on flooding and erosion during extreme storm events.}

\cventry{2013 -- 2014}{Graduate intern at the department of Sediment transport and Morphology}{\textsc{Deltares}}{Delft, the Netherlands}{\footnotesize{\url{http://www.deltares.nl}}}{A research study addressing the formation of coral cays on platform reefs.}


\subsection{Sea-going experience}
\cventry{2018}{Principal student investigator}{\textsc{RV Investigator}}{Polar Front meander downstream of the South East Indian Ridge}{16 October -- 16 November 2018}{Assisting in the planning of CTD locations along 11 transects, operating CTD casts/ sampling Niskin bottles and analysing the incoming data. Deploying EM-APEX floats and Langrangian surface drifters.}

\cventry{2015}{Investigator}{\textsc{RV Pelagia}}{Rainbow thermal field on the Mid-Atlantic Ridge near the Azores}{6 - 14 April 2015}{Assisting in recovering deep-sea moorings, controlling CTD casts during "yo-yo" and "tow-yow" surveys and a first attempt in analysing the recovered mooring datasets.}

\cventry{2008 -- 2015}{Seaman at the deck department}{\textsc{Rederij Halfland} - shipping company}{Rotterdam, the Netherlands}{\footnotesize{\url{http://www.halfland.nl}}}{Assisting steersman and captain on two and three-masted sailing ships.}

\subsection{Teaching experience}
\cventry{2017}{Teacher in Physics}{\textsc{Lyceo}}{Leiden, the Netherlands}{\footnotesize{\url{http://www.lyceo.nl}}}{Giving instruction and additional attention to secondary school students.}

\cventry{2015}{Teacher in Physics and Mathematics}{\textsc{Lyceo/ Inwijs}}{Leiden, the Netherlands}{\footnotesize{\url{http://www.lyceo.nl}} / \footnotesize{\url{http://www.inwijs.nl}}}{Giving instruction and additional attention to secondary school students.}


\subsection{Engineering experience}
\cventry{2011}{Undergraduate intern at the department of Research and Development}{\textsc{CTC GmbH} subsidiary of \textsc{Airbus Deutschland GmbH}}{Stade, Germany}{\footnotesize{\url{http://www.ctc-gmbh.com}}}{Developing processes for "Out of Autoclave" rework on the Airbus A350-1000 Fuselage, made primarily of carbon fibre reinforced plastic.}

\cventry{2007 -- 2008}{Intern at the department of Design/ Analysis and Project Management}{\textsc{FACC AG}}{Ried im Innkreis, Austria}{\footnotesize{\url{http://www.facc.at}}}{Helping in the design and analysis of composite components for the Boeing 787 Spoiler project.}

\section{Summer/ winter schools/ educational programs}
\cventry{2022}{Ocean Impact Ideation Program}{\textsc{Ocean Impact Organisation}}{Australia}{}{}
\cventry{2020}{Virtual winter school in Atmosphere and Ocean Dynamics}{\textsc{ARC Centre of Excellence for Climate Extremes}}{Australia}{}{}
\cventry{2018}{Winter school in Climate Extremes and High Impact Weather}{\textsc{Australian National University}}{Canberra, Australia}{}{}
\cventry{2011}{Summer course in Multidisciplinary sea-going oceanography}{\textsc{Royal Netherlands Institute for Sea Research (NIOZ)}}{Texel, the Netherlands}{}{}
\cventry{2011}{Summer school in Physics of the Climate System}{\textsc{Utrecht University}}{Utrecht, the Netherlands}{}{}

%----------------------------------------------------------------------------------------
%	PUBLICATIONS
%----------------------------------------------------------------------------------------
\section{Publications}
\nocite{*}
%\bibliographystyle{plain}
\bibliographystyle{plainyrrev}
\bibliography{publications} % 'publications' is the name of a BibTeX file

%----------------------------------------------------------------------------------------
%	PRESENTATIONS
%----------------------------------------------------------------------------------------

\section{Conference presentations}
\cventry{2022}{Deep cyclogenesis beneath a standing meander in the Subantarctic Front}{AMOS Annual Conference}{Adelaide, Australia}{November 28-December 1, 2022}{}

\cventry{2022}{Meander dynamics in the Antarctic Circumpolar Current}{FilaChange}{Hobart, Australia}{August 29-September 2, 2022}{}

\cventry{2022}{Dynamics of a standing meander of the Subantarctic Front diagnosed from satellite altimetry and along-stream anomalies of temperature and salinity}{Ocean Sciences Meeting}{Honolulu, HI, USA (virtually)}{February 24-March 4, 2022}{}

\cventry{2022}{Dynamics of a standing meander of the Subantarctic Front diagnosed from satellite altimetry and along-stream anomalies of temperature and salinity}{International Conference on Southern Hemisphere Meteorology and Oceanography (ICSHMO)}{Christchurch, New Zealand (virtually)}{February 8-12, 2022}{}

\cventry{2019}{Sustainability in ocean observations}{OceanObs}{Honolulu, HI, USA}{September 16-20, 2019}{}

\cventry{2019}{Three-dimensional structure of a standing meander in the Antarctic Circumpolar Current}{EGU General Assembly}{Vienna, Austria}{April 7-12, 2019}{}

\cventry{2018}{Are standing meanders braking the Antarctic Circumpolar Current?}{Annual Graduate Research Conference organised by the \textsc{University of Tasmania}}{Hobart, Australia}{September 07, 2018}{}

\cventry{2017}{Integrated approaches to coastal risk - case study: Kristianstad Municipality, Sweden.}{Final Conference organised by the \href{http://www.risckit.eu}{RISC-KIT} Consortium and hosted at \textsc{Deltares}}{Delft, the Netherlands}{April 5-7, 2017}{}

\cventry{2016}{From storm hazards at the coast, to impacts on the hinterland - \href{http://www.risckit.eu}{RISC-KIT} project}{Climate Change Adaptation in the Coastal Zone conference organised by \textsc{F\"{o}reningen Vatten} (Swedish Association for Water) in cooperation with the \textsc{Lund University} and the \textsc{World Maritime University}}{Malm\"{o}, Sweden}{April 5-6, 2016}{}

%----------------------------------------------------------------------------------------
%	AWARDS AMD GRANTS
%----------------------------------------------------------------------------------------

\section{Grants}
\cventry{2022}{Write-up Scholarship}{\textsc{CSIRO-UTAS PhD Program in Quantitative Marine Science}}{Australia}{AUD\$8,000}{}
\cventry{2019}{Travel support for OceanObs'19}{\textsc{Centre of Excellence for Climate Extremes}}{Australia}{AUD\$3,000}{}
\cventry{2018}{Travel support for research visit to WHOI/ University of Rhode Island}{\textsc{ARC Discovery Grant (DP170102162-Bindoff)}}{Australia}{AUD\$4,000}{}
\cventry{2018}{Travel support for research visit to WHOI}{\textsc{Centre of Excellence for Climate Extremes}}{Australia}{AUD\$1,000}{}
\cventry{2018}{Quantitative Marine Science postgraduate conference support}{\textsc{CSIRO} and \textsc{University of Tasmania}}{Hobart, Australia}{AUD\$2,500}{}
\cventry{2017}{Quantitative Marine Science Scholarship}{\textsc{CSIRO} and \textsc{University of Tasmania}}{Hobart, Australia}{top-up of AUD\$5,000pa for three years with a possible six month extension}{}
\cventry{2017}{Tasmania Graduate Research Scholarship}{\textsc{University of Tasmania}}{Hobart, Australia}{living allowance of AUD\$26,682pa for three years with a possible six month extension}{}


%----------------------------------------------------------------------------------------
%	LANGUAGES SECTION
%----------------------------------------------------------------------------------------

\section{Languages}
\cvlanguage{Dutch}{Native language}{}
\cvlanguage{English}{Fluent}{speaking and writing}
\cvlanguage{German}{Intermediate}{speaking and writing}
\cvlanguage{French}{Basic knowledge}{speaking and writing}
\cvlanguage{Portuguese}{Basic knowledge}{speaking}
\cvlanguage{Spanish}{Basic knowledge}{speaking}

%----------------------------------------------------------------------------------------
%	SKILLS/ ABILITIES SECTION
%----------------------------------------------------------------------------------------

\section{Skills/ Abilities}

\subsection{Communication}
\cvitem{Reading}{Proficient in reading technical manuals and scientific papers}
\cvitem{Writing}{Proficient in writing technical manuals and scientific papers}
\cvitem{Presenting}{Confident in public speaking and work floor discussions}
\cvitem{Outreach}{STEM\_next at Taroona High School (2019), Roving Scientist on Beaker Street at the Tasmanian Museum and Art Gallery (2019), IN1810v05 Voyage blog post (2018)}

\subsection{Leadership/ teamwork}
\cventry{2019 -- 2020}{Chair of the Early Career Researcher committee}{\textsc{Centre of Excellence for Climate Extremes}}{Australia}{\footnotesize{\url{https://climateextremes.org.au}}}{}
\cventry{2018 -- 2020}{Member of the Early Career Researcher committee}{\textsc{Centre of Excellence for Climate Extremes}}{Australia}{\footnotesize{\url{https://climateextremes.org.au}}}{}
\cventry{2019}{Co-organiser of CSIRO-IMAS ECR-day}{\textsc{CSIRO} and \textsc{IMAS}}{Hobart, Australia}{}{}
\cventry{2018}{Student representative for the Earth Systems Science Compute User Group}{\textsc{University of Tasmania}}{Hobart, Australia}{}{}

% \subsection{Computation/ numeracy}

\subsection{Information technology}
\cvcomputer{Operating systems}{Linux (preferred), ability to work cross-platform}
\cvcomputer{Programming languages}{Python, shell scripting, Matlab, Fortran, HTML/CSS}
\cvcomputer{Productivity software}{LibreOffice, LaTeX, Microsoft Office suite}
\cvcomputer{High-performance/ cloud computing}{NCI High Performance Computing system, Deltares Cluster Computing System, SEI work station}
\cvcomputer{Open-source tools and documentation}{Conda, Git, Markdown, readthedocs.io}
\cvcomputer{Engineering design packages}{CATIA, Patran/ Nastran, Matlab/ Simulink}
\cvcomputer{Modelling software}{XBeach, Delft3D, LISFLOOD}
\cvcomputer{Geographic information system}{QGIS}

\subsection{Licences/ Certificates}
\cvitemwithcomment{Date of expiry: \newline 3 March 2056}{International certificate for operators of pleasure craft}{Inland/ Coastal waters}
\cvitemwithcomment{}{Long Range Operators Certificate of Proficiency}{}
\cvitemwithcomment{26 March 2020}{Survival at Sea certificate}{}
\cvitemwithcomment{3 March 2017}{Seafarer medical certificate}{Worldwide}
\cvitemwithcomment{30 January 2019}{Car and motorcycle driving license}{category A and B}

%----------------------------------------------------------------------------------------
%	MEMBERSHIPS
%----------------------------------------------------------------------------------------

%\section{Memberships}
%\cvitem{S.W.V. Plankenkoorts}{Member of students windsurf club}
%\cvlistitem{Member of activities committee}
%\cvlistitem{Organisation member of the band contest "MadFest" and the event "IJsselmeer Crossing"}
%\cvitem{Scouting Cycloongroup Borculo}{Formerly member}
%\cvlistitem{Giving guidance to scouts of 10 to 15 years old.}
%\cvlistitem{Chair of our own Scout group.}
%\cvitem{K.S.V. Sanctus Virgilius}{Formerly member of students' union}


%\cventry{2007 -- Present}{\normalfont{Member of students windsurf club}}{S.W.V. Plankenkoorts}{Delft, the Netherlands}{}{\begin{itemize}
%\item Member of activities committee
%\item Organisation member of the band contest "MadFest" and the event "IJsselmeer Crossing"\end{itemize}}
%\cventry{2006 -- 2010}{\normalfont{Formerly member of students' union}}{K.S.V. Sanctus Virgilius}{Delft, the Netherlands}{}{}
%\cventry{1994 -- 2002}{\normalfont{Formerly member of Scouting}}{Cycloongroep Borculo}{Borculo, the Netherlands}{}{\begin{itemize}
%\item Giving guidance to scouts of 10 to 15 years old.
%\item President of our own Scout group.
%\end{itemize}}

%----------------------------------------------------------------------------------------
%	INTERESTS SECTION
%----------------------------------------------------------------------------------------

% \section{Other interests}
% \cvlistdoubleitem{Surfing}{Motorcycling}
% \cvlistdoubleitem{Sailing}{Restoring an English classic sports car}
% \cvlistitem{Travelling (South America (6 months in 2014), Panama, Costa Rica, Brazil, Indonesia, Fuerteventura and Europe)}
%\cvlistdoubleitem{Snowboarding}{Capoeira}
%\cvlistdoubleitem{Speed skating}{Scuba diving}


\section{References}
\subsection{\normalfont{Available upon request}}

% \begin{cvcolumns}
% 	\cvcolumn{}{%
% 		\cvreference{A/Prof. Dr. H.E. Phillips}{Associate Professor}{Institute for Marine and Antarctic Studies/ University of Tasmania}{}{}{}{h.e.phillips@utas.edu.au}{}{\href{tel:0061362262994}{+61 3 6226 2994}}\\
% 	}
% 	\cvcolumn{}{%
% 		\cvreference{Prof. Dr. N.L. Bindoff}{Australian Antarctic Program Partnership - Program Leader}{Institute for Marine and Antarctic Studies/ University of Tasmania}{}{}{}{n.bindoff@utas.edu.au}{}{\href{tel:0061362262986}{+61 3 6226 2986}}\\
% 	}
% 	\cvcolumn{}{%
% 		\cvreference{Dr. S.R. Rintoul}{CSIRO Fellow and Research Team Leader}{CSIRO Oceans and Atmosphere}{}{}{}{steve.rintoul@csiro.au}{}{\href{tel:0061362325393}{+61 3 6232 5393}}\\
% 	}
% 	\cvcolumn{}{%
% 		\cvreference{A/Prof. Dr. ir. A. van Dongeren}{Specialist in nearshore wave dynamics and morphology}{Deltares}{}{}{}{ap.vandongeren@deltares.nl}{}{\href{tel:0031152858585}{+31 (0)15 285 8585}}\\
% 	}
% \end{cvcolumns}

%\begin{cvcolumns}
%	\cvcolumn{}{%
%		\cvreference{Dr. K. Barquet}{Research Fellow\\}{Stockholm Environment Institute}{}{}{}{karina.barquet@sei-international.org}{}{\href{tel:0046703885690}{+46 (0)70 388 5690}}\\
%		%		\cvreference{Drs. A. Degeling}{Owner and director}{Rederij Halfland}{Leuvehaven 68}{3011 EA Rotterdam, the Netherlands}{}{rederij@halfland.nl}{+31 (0)10 414 6728}{+31 (0)6 227 87 207}
%	}
%	\cvcolumn{}{%
%		\cvreference{Dr. ir. A. van Dongeren}{Specialist in nearshore wave dynamics and morphology}{Deltares}{}{}{}{ap.vandongeren@deltares.nl}{}{\href{tel:0031614702455}{+31 (0)6 147 02 455}}\\
%	}
%\end{cvcolumns}


%\begin{cvcolumns}
%	\cvcolumn{}{%
%		\cvreference{Prof. dr. H.E. de Swart}{Professor in coastal and shelf sea dynamics}{IMAU - Institute for Marine and \\Atmospheric research Utrecht}{\textbf{Postal address}}{P.O. Box 80.005}{3508 TA Utrecht, the Netherlands}{h.e.deswart@uu.nl}{+31 (0)30 253 3275}{}\\
%		\cvreference{Drs. A. Degeling}{Owner and director}{Rederij Halfland}{Leuvehaven 68}{3011 EA Rotterdam, the Netherlands}{}{rederij@halfland.nl}{+31 (0)10 414 6728}{+31 (0)6 227 87 207}
%	}
%	\cvcolumn{}{%
%		\cvreference{Dr. ir. A. van Dongeren}{Specialist in nearshore wave dynamics and morphology}{Deltares - institute for applied research}{Rotterdamseweg 185}{2629 HD Delft, the Netherlands}{}{ap.vandongeren@deltares.nl}{}{+31 (0)6 147 02 455}\\
%	}
%	
%\end{cvcolumns}

\end{document}